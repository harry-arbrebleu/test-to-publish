\documentclass{ltjsarticle}
\usepackage{luatexja}
\usepackage{amsmath, amssymb, amsfonts, latexsym, bm, amsthm, url}
\usepackage{luatexja-fontspec, ascmac, fancybox, pdfpages}
\usepackage{tabularx, footmisc, colortbl, here, usebib, microtype, listings, jvlisting}
\usepackage{graphicx, luacode, caption, fancyhdr, fancybox, color}
\usepackage[top = 20truemm, bottom = 20truemm, left = 20truemm, right = 20truemm]{geometry}
\usepackage{enumitem}
\everymath{\displaystyle}
\renewcommand*\oldstylenums[1]{\textosf{#1}}
\renewcommand*\oldstylenums[1]{\textosf{#1}}
\setmainfont[Ligatures=TeX]{ShipporiMincho-OTF-Regular}
\setsansfont[Ligatures=TeX]{ShipporiMincho-OTF-Regular}
\setmainjfont[BoldFont=ShipporiMincho-OTF-Bold]{ShipporiMincho-OTF-Regular}
\setsansjfont{ShipporiMincho-OTF-Regular}
\newjfontfamily\jisninety[CJKShape=JIS1990]{ShipporiMincho-OTF-Bold}
\newcolumntype{t}{!{\vrule width 0.1pt}} % 細い縦線の定義
\newcolumntype{b}{!{\vrule width 1.5pt}} % 太い縦線の定義
\renewcommand{\theenumi}{(\roman{enumi})}
\renewcommand{\r}[1]{\mathrm{#1}}
\renewcommand{\c}{\si{\degreeCelsius}}
\renewcommand{\d}{\r{d}}
\renewcommand{\phi}{\varphi}
\renewcommand{\epsilon}{\varepsilon}
\renewcommand{\thefootnote}{\fnsymbol{footnote}}
\renewcommand{\theequation}{\thesubsection.\arabic{equation}}
\renewcommand{\footrulewidth}{0.4pt}
\newcommand{\bb}[1]{\mathbb{#1}}
\newcommand{\fig}[3]{
  \begin{figure}[H]
    \centering
    \includegraphics[width = 100mm]{#1}
    \caption{#2} \label{#3}
  \end{figure}
}
\newcommand{\reff}[1]{図\ref{#1}}
\newcommand{\reft}[1]{表\ref{#1}}
\newcommand{\mar}[1]{\textcircled{\scriptsize #1}}
\newcommand{\combination}[2]{{}_{#1} \mathrm{C}_{#2}}
\newcommand{\thline}{\noalign{\hrule height 0.1pt}} % 細い横線の正義
\newcommand{\bhline}{\noalign{\hrule height 1.5pt}} % 太い横線の定義
\setcounter{tocdepth}{4}
% \mathtoolsset{showonlyrefs = true}
\makeatletter
\@addtoreset{equation}{subsubsection}
\makeatother
\lstset{
  basicstyle={\ttfamily},
  identifierstyle={\small},
  commentstyle={\smallitshape},
  keywordstyle={\small\bfseries},
  ndkeywordstyle={\small},
  stringstyle={\small\ttfamily},
  frame={tb},
  breaklines=true,
  columns=[l]{fullflexible},
  numbers=left,
  xrightmargin=0mm,
  xleftmargin=3mm,
  numberstyle={\scriptsize},
  stepnumber=1,
  numbersep=1mm,
  lineskip=-0.5ex
}
\title{線型代数}
\author{hoge}
\author{Chef: 佐々木\ 優真(momoyuu)}
\pagestyle{empty}
\date{}

\begin{document}
  \maketitle
    \section{概要}
    \begin{center}
      \begin{description}
        \item[サブタイトル: ] 量子論の本質的理解
        \item[曜日/時限: ] 150分を1コマとする.休憩を2回挟む.
        \item[教室: ] 未定(主として対面だが,VCなどを用いることもある)
        \item[指定図書: ] 清水明,「量子論の基礎」,サイエンス社
        \item[目的: ] 計算に埋もれがち量子論の理論体系を学んでから具体例に親しむ.
        \item[前提知識: ] 線型代数の計算方法.
      \end{description}
    \end{center}
    \section{計画}
      \begin{enumerate}
        \item 第1章: 「古典力学の破綻」,第2章: 「基本的な枠組み」
        \item 第3章: 「閉じた有限自由度系の純粋状態の量子論」(1節: 「基本的な考え方」から8節: 「ブラとケット」まで)
        \item 第3章: 「閉じた有限自由度系の純粋状態の量子論」(9節: 「射影演算子」から16節: 「連続固有値に関する数学的注意」まで)
        \item 第3章: 「閉じた有限自由度系の純粋状態の量子論」(17節: 「ボルンの確率規則―連続固有値の場合」から26節: 「射影仮説について」まで)
        \item 第4章: 「有限自由度系の正準量子化」(1節: 「古典解析力学」から3節: 「正準交換関係のシュレディンガー表現」まで)
        \item 第4章: 「有限自由度系の正準量子化」(4節: 「フォン・ノイマンの一意性定理」から「無限次元ヒルベルト空間の注意」まで)
        \item 第5章: 「1次元空間を運動する粒子の量子論」(1節: 「1次元空間を運動する粒子のシュレディンガー方程式」から9節: 「重ね合わせの例」まで)
        \item 第5章: 「1次元空間を運動する粒子の量子論」(10節: 「不確定原理のよる基底準位の見積もり」から16節: 「調和振動子」まで)
        \item 第6章: 「時間発展について」,第7章: 「場の量子論―場の量子論入門」
        \item 第8章: 「ベルの不等式」,第9章: 「基本変数による記述のまとめ」
      \end{enumerate}
    \section{関連ゼミ}
      \begin{description}
        \item[解析力学: ] 担当者: 増田\ 悠人(y.masu)
        \item[関数解析: ] 担当者: 大黒\ 瑠海空(luke\_0404)
        \item[線型代数: ] 担当者: 佐々木\ 優真(momoyuu)
        \item[実解析: ] 担当者: 大黒\ 瑠海空(luke\_0404)
      \end{description}
\end{document}